% !TeX spellcheck = de_DE

\chapter{Objectives and Work Program}

\section{Objectives}
\label{sec-objectives}

This thesis aims to improve and investigate the following:

\begin{itemize}
	\item In \emph{one field}, the focus is on \emph{something}, while \emph{problems are these}, where yet undiscovered defects may loom.
	\item In \emph{other field}, the goal is to \emph{do this}, unveiling new horizons.
	\item In \emph{the third}, the idea is to achieve excellent achievement in the field of excellence.
\end{itemize}

\noindent The key concept for these aims is \emph{this}, a  technique which does that.  These techniques are \emph{normal} in contrast to \emph{abnormal} behavior, and thus fuel the innovations above.

%\noindent The project is structured into four work packages, whose interplay is summarized in the following. I will begin with something:
%
%\begin{description}
%	\item[\ref{wp:something}: Something something and doing something.] The first work package (\ref{wp:something}) will provide something. A blind text like this gives you information about the selected font, how the letters are written and the impression of the look.
%\end{description}
%\noindent
%Problems from \ref{wp:something} will then form a base for next three work packages (\ref{wp:work-something}, \ref{wp:do-this} and \ref{wp:do-that}).
%\begin{description}
%	\item[\ref{wp:work-something}: Working in a coal mine.] The second work package will use something to do the work on doing the things.
%	\item[\ref{wp:do-this}: Sixteen tons.]  Using white-box approach, dirt will be dug.
%	\item[\ref{wp:do-that}: Celebration preparation.]  In the fourth work package, using data provided by \ref{wp:work-something}, we prepare the celebration party. If this proposal gets accepted.
%\end{description}
%
%\noindent
%In all work packages, the research shall be driven by \emph{some procedure} on real-life systems or dead-life systems.  Furthermore, the results from WP2--WP4 will continuously provide feedback for WP1. This is basically a lot of nonsensical text.


\section{Work Program}
\label{sec-workprogram}
%gant
% In this section, sections are organized by packages -- AZ
\newcounter{wp}
\let\oldthesubsection=\thesubsection
\def\thesubsection{WP\arabic{wp}}
% \def\package#1{\subsection[\protect\thesubsection{} {#1}]{#1}}
\def\package#1{\addtocounter{wp}{1}\subsection{#1}}

% Use \task to introduce tasks
\let\oldthesubsubsection=\thesubsubsection
\def\thesubsubsection{\thesubsection.\arabic{subsubsection}}
% \def\task#1{\subsubsection[\thesubsubsection{} {#1}]{#1}}
\let\task=\subsubsection

To deliver the proposed research this thesis, the following research plan has been derived from the research tasks of the individual work packages WP1-4. The individual tasks are aligned with each other and intermediate results will be exchanged.

\package{Something something and doing something.}
\label{wp:something}

\paragraph{Challenges and motivation.}
This text should show, how a printed text will look like at this place. If you read this text, you will get no information. Really? Is there no information? Is there a difference between this text and some nonsense like Huardest gefburn. Kjift – Never mind! A blind text like this gives you information about the
selected font, how the letters are written and the impression of the look.

\noindent For this work package, the main challenge will be to find the right balance between something and something else.  This search will be driven by another things and experiments in \ref{wp:work-something}, \ref{wp:do-this}, \ref{wp:do-that}.

\paragraph{Research plan and individual research tasks.}
In this work package, we create the theoretical foundation for the required somethings that satisfy the requirements of the something else.

\task{Refining something}
\label{task-wp1-1}

The first task aims to provide effective algorithms for probabilistic model construction.  Existing approaches~\cite{Hoorn2012} will be used as a baseline and derive algorithms that do something. I am now just citing~\cite{AspectJ}.  Novel probabilistic something doings are expected to
\begin{compact_itemize}
    \item do something that will be explained here in a sentence or two
    \item do something else in this using {\it ReallyNiceFramework}~\cite{Frey2011}.
\end{compact_itemize}
%
For this, we will have to deal with two challenges:
\begin{description}
    \item[Noise.] This text should show, how a printed text will look like at this place. If you read this \emph{text}, you will get no information. \emph{Really?} Is there no information? Is there a difference between this text and some nonsense like Huardest gefburn. Kjift – Never mind! A blind text like this gives you information about the selected font, how the letters are written and the impression of the look

Look at this wonderful paragraph with mathematical formula.  Rather than learning a complete model that covers the entire state space, axioms that capture \emph{conditional probabilities} between two or more events may already be sufficient for producing test suites with comparable quality.  For instance, if we know that two events $A$~and~$B$ normally follow each other (formally: $\mathcal{P}_{\geq 1} [\square(A \Rightarrow \mathcal{P}_{\geq 0.99}(\Diamond B) )]$), we can focus on precisely such sequences (to test common behavior), or its complement (to test uncommon behavior).

    \item[Granularity.]  We would be na\"\i{}ve to expect a single ``one size fits all'' specification. At vero eos et accusam et justo duo dolores et ea rebum. Stet clita kasd gubergren, no sea takimata sanctus est Lorem ipsum dolor sit amet.

Ut wisi enim ad minim veniam, quis nostrud exerci tation ullamcorper suscipit lobortis nisl ut aliquip ex ea commodo consequat. Duis autem vel eum iriure dolor in hendrerit in vulputate velit esse molestie consequat, vel illum dolore eu feugiat nulla facilisis at vero eros et accumsan et iusto odio dignissim qui blandit praesent luptatum zzril delenit augue duis.
\end{description}

\noindent It is thus expected that this WP be driven by the impact of algorithmic choices on the effectiveness of the mined specifications in tasks \ref{wp:do-this}, \ref{wp:do-that}, and~\ref{wp:work-something}.  After an experimental evaluation on a large set of sample instances, each iteration of the WP will thus produce individual software assessment tasks.

\task{Think of something}
\label{task-wp1-2}

Do something similar in task 2.

\package{Working in a coal mine}
\label{wp:work-something}
Again, describe what you will do in this package, refine it in subtasks.\\
Ut wisi enim ad minim veniam, quis nostrud exerci tation ullamcorper suscipit lobortis nisl ut aliquip ex ea commodo consequat. Duis autem vel eum iriure dolor in hendrerit in vulputate velit esse molestie consequat, vel illum dolore eu feugiat nulla facilisis at vero eros et accumsan et iusto odio dignissim qui blandit praesent luptatum zzril delenit augue duis

\package{Sixteen tons.}
\label{wp:do-this}
Again, describe what you will do in this package, refine it in subtasks.\\
Ut wisi enim ad minim veniam, quis nostrud exerci tation ullamcorper suscipit lobortis nisl ut aliquip ex ea commodo consequat. Duis autem vel eum iriure dolor in hendrerit in vulputate velit esse molestie consequat, vel illum dolore eu feugiat nulla facilisis at vero eros et accumsan et iusto odio dignissim qui blandit praesent luptatum zzril delenit augue duis


\package{Celebration preparation}
\label{wp:do-that}
Again, describe what you will do in this package, refine it in subtasks.\\
Ut wisi enim ad minim veniam, quis nostrud exerci tation ullamcorper suscipit lobortis nisl ut aliquip ex ea commodo consequat. Duis autem vel eum iriure dolor in hendrerit in vulputate velit esse molestie consequat, vel illum dolore eu feugiat nulla facilisis at vero eros et accumsan et iusto odio dignissim qui blandit praesent luptatum zzril delenit augue duis

%\begin{figure}
%	\begin{center}
%
%		\begin{ganttchart}[y unit title=0.4cm,
%				y unit chart=0.5cm,
%				vgrid,hgrid, 
%				title label anchor/.style={below=-1.6ex},
%				title left shift=.05,
%				title right shift=-.05,
%				title height=1,
%				bar/.style={fill=gray!50},
%				incomplete/.style={fill=white},
%				progress label text={},
%				bar height=0.7,
%				group right shift=0,
%				group top shift=.6,
%				group height=.3,
%			group peaks={}{}{.2}]{24}
%%labels
%			\gantttitle{Week}{24} \\
%			\gantttitle{Monday}{4} 
%			\gantttitle{Tuesday}{4} 
%			\gantttitle{Wednesday}{4} 
%			\gantttitle{Thursday}{4} 
%			\gantttitle{Friday}{4} 
%			\gantttitle{Saturday}{4} \\
%%tasks
%			\ganttbar{first task}{1}{2} \\
%			\ganttbar{task 2}{3}{8} \\
%			\ganttbar{task 3}{9}{10} \\
%			\ganttbar{task 4}{11}{15} \\
%			\ganttbar[progress=33]{task 5}{20}{22} \\
%			\ganttbar{task 6}{18}{19} \\
%			\ganttbar{task 7}{16}{18} \\
%			\ganttbar[progress=0]{task 8}{21}{24}
%
%%relations 
%			\ganttlink{elem0}{elem1} 
%			\ganttlink{elem0}{elem3} 
%			\ganttlink{elem1}{elem2} 
%			\ganttlink{elem3}{elem4} 
%			\ganttlink{elem1}{elem5} 
%			\ganttlink{elem3}{elem5} 
%			\ganttlink{elem2}{elem6} 
%			\ganttlink{elem3}{elem6} 
%			\ganttlink{elem5}{elem7} 
%		\end{ganttchart}
%	\end{center}
%	\caption{Gantt Chart}
%\end{figure}
